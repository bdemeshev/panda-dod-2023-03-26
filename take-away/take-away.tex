% arara: xelatex
\documentclass[12pt]{article}

\usepackage{physics}


\usepackage{tikz} % картинки в tikz
\usepackage{microtype} % свешивание пунктуации

\usepackage{array} % для столбцов фиксированной ширины

\usepackage{indentfirst} % отступ в первом параграфе

\usepackage{sectsty} % для центрирования названий частей
\allsectionsfont{\centering}

\usepackage{amsmath, amsfonts, amssymb} % куча стандартных математических плюшек

\usepackage{comment}

\usepackage[top=2cm, left=1.2cm, right=1.2cm, bottom=2cm]{geometry} % размер текста на странице

\usepackage{lastpage} % чтобы узнать номер последней страницы

\usepackage{enumitem} % дополнительные плюшки для списков
%  например \begin{enumerate}[resume] позволяет продолжить нумерацию в новом списке
\usepackage{caption}

\usepackage{url} % to use \url{link to web}

\usepackage{fancyhdr} % весёлые колонтитулы
\pagestyle{fancy}
\lhead{Прикладной анализ данных}
\chead{}
\rhead{ДоД, 2023-03-26}
\lfoot{}
\cfoot{НЕ ПАНИКОВАТЬ!}
\rfoot{}
\renewcommand{\headrulewidth}{0.4pt}
\renewcommand{\footrulewidth}{0.4pt}

\usepackage{tcolorbox} % рамочки!

\usepackage{todonotes} % для вставки в документ заметок о том, что осталось сделать
% \todo{Здесь надо коэффициенты исправить}
% \missingfigure{Здесь будет Последний день Помпеи}
% \listoftodos - печатает все поставленные \todo'шки


% более красивые таблицы
\usepackage{booktabs}
% заповеди из докупентации:
% 1. Не используйте вертикальные линни
% 2. Не используйте двойные линии
% 3. Единицы измерения - в шапку таблицы
% 4. Не сокращайте .1 вместо 0.1
% 5. Повторяющееся значение повторяйте, а не говорите "то же"



\usepackage{fontspec}
\usepackage{polyglossia}

\setmainlanguage{russian}
\setotherlanguages{english}

% download "Linux Libertine" fonts:
% http://www.linuxlibertine.org/index.php?id=91&L=1
\setmainfont{Linux Libertine O} % or Helvetica, Arial, Cambria
% why do we need \newfontfamily:
% http://tex.stackexchange.com/questions/91507/
\newfontfamily{\cyrillicfonttt}{Linux Libertine O}

\AddEnumerateCounter{\asbuk}{\russian@alph}{щ} % для списков с русскими буквами
\setlist[enumerate, 2]{label=\asbuk*),ref=\asbuk*}

%% эконометрические сокращения
\DeclareMathOperator{\Cov}{\mathbb{C}ov}
\DeclareMathOperator{\Corr}{\mathbb{C}orr}
\DeclareMathOperator{\Var}{\mathbb{V}ar}

\let\P\relax
\DeclareMathOperator{\P}{\mathbb{P}}

\DeclareMathOperator{\E}{\mathbb{E}}
% \DeclareMathOperator{\tr}{trace}
\DeclareMathOperator{\card}{card}
\DeclareMathOperator{\plim}{plim}
\DeclareMathOperator{\pCorr}{\mathrm{p}\mathbb{C}\mathrm{orr}}


\newcommand \hb{\hat{\beta}}
\newcommand \hs{\hat{\sigma}}
\newcommand \htheta{\hat{\theta}}
\newcommand \s{\sigma}
\newcommand \hy{\hat{y}}
\newcommand \hY{\hat{Y}}
\newcommand \e{\varepsilon}
\newcommand \he{\hat{\e}}
\newcommand \z{z}
\newcommand \hVar{\widehat{\Var}}
\newcommand \hCorr{\widehat{\Corr}}
\newcommand \hCov{\widehat{\Cov}}
\newcommand \cN{\mathcal{N}}
\newcommand \RR{\mathbb{R}}
\newcommand \NN{\mathbb{N}}
\newcommand{\cF}{\mathcal{F}}
\newcommand{\cH}{\mathcal{H}}


\begin{document}

\begin{enumerate}



\item Города $A$ и $B$ соединены железной дорогой. 
Поезда в обе стороны отправляются из них каждый час одновременно, время в пути составляет ровно час. 
Стрелочник, живущий в домике при железной дороге, любит подойти к окну в случайный момент времени,
дождаться первого проходящего мимо поезда и записать его направление. 
Поезда обоих направлений в его записях встречаются одинаково часто.
\begin{enumerate}
    \item В скольки минут пути на поезде от ближайшего города он живёт?
    \item Сколько в среднем он ждёт поезда?
\end{enumerate}

\item Аня хватается за верёвку в форме окружности в произвольной точке. 
Боря берёт мачете и с завязанными глазами разрубает верёвку в двух случайных независимых местах. 
Аня забирает себе тот кусок, за который держится. Боря забирает оставшийся кусок. 
Вся верёвка имеет единичную длину.

Чему равна вероятность того, что у Ани верёвка окажется длиннее?

\item Илье Муромцу предстоит дорога к камню. От камня начинаются ещё три дороги. Каждая
из тех дорог снова оканчивается камнем. И от каждого камня начинаются ещё три дороги.
И каждые те три дороги оканчиваются камнем...{ }
И так далее до бесконечности. 
На каждой дороге живёт трёхголовый Змей Горыныч. Каждый Змей Горыныч бодрствует независимо
от других с вероятностью одна третья. 

У Василисы Премудрой существует Чудо-Карта, на которой видно, какие Змеи Горынычи бодрствуют, а какие — нет.

\begin{enumerate}
    \item Какова вероятность того, что Илья Муромец будет идти только мимо спящих Змеев Горынычей, 
    если будет всегда выбирать дороги наугад?
    \item Какова вероятность того, что Василиса Премудрая \textit{сможет найти 
    на карте} путь, проходящий всегда мимо спящих Змеев Горынычей?
\end{enumerate}


\end{enumerate}


\end{document}

